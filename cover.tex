\frontmatter 
% cabeçalho para as páginas das seções anteriores ao capítulo 1 (frontmatter)
\fancyhead[RO]{{\footnotesize\rightmark}\hspace{2em}\thepage}
\setcounter{tocdepth}{2}
\fancyhead[LE]{\thepage\hspace{2em}\footnotesize{\leftmark}}
\fancyhead[RE,LO]{}
\fancyhead[RO]{{\footnotesize\rightmark}\hspace{2em}\thepage}

\onehalfspacing  % espaçamento

% ---------------------------------------------------------------------------- %
% CAPA
% Nota: O título para as dissertações/teses do IME-USP devem caber em um 
% orifício de 10,7cm de largura x 6,0cm de altura que há na capa fornecida pela SPG.
\thispagestyle{empty}
\begin{center}
	\vspace*{2.3cm}
	
	\textbf{\Large{
		\br{Título do trabalho a ser apresentado à\\CPG para a dissertação/tese}
		\en{Paper Title to be delivered to\\CPG to dissertation/thesis}
	}}\\
	
	\vspace*{1.2cm}
	\Large{Nome completo do Autor}
	
	\vskip 2cm
	\textsc{
		\br{
			Dissertação/Tese apresentada\\[-0.25cm] 
			ao\\[-0.25cm]
			Instituto de Matemática e Estatística\\[-0.25cm]
			da\\[-0.25cm]
			Universidade de São Paulo\\[-0.25cm]
			para\\[-0.25cm]
			obtenção do título\\[-0.25cm]
			de\\[-0.25cm]
			Mestre/Doutor em Ciências
			}
		\en{
			Dissertation/Thesis presented\\[-0.25cm] 
			to\\[-0.25cm]
			Institute of Mathematics and Statistics\\[-0.25cm]
			from\\[-0.25cm]
			University of Sao Paulo\\[-0.25cm]
			to\\[-0.25cm]
			get the title\\[-0.25cm]
			of\\[-0.25cm]
			Master/Doctor of Science
		}
	}
	
	\vskip 1.5cm
	\br{
		Programa: Nome do Programa\\
		Orientador: Prof. Dr. Nome do Orientador\\
		Coorientador: Prof. Dr. Nome do Coorientador
	}
	\en{
		Program: Program Name\\
		Advisor: Prof. Dr. Nome do Orientador\\
		Coadvisor: Prof. Dr. Nome do Coorientador
	}
	
	
	\vskip 1cm
	\normalsize{
		\br{Durante o desenvolvimento deste trabalho o autor recebeu auxílio
			financeiro da CAPES/CNPq/FAPESP}
		\en{Durante o desenvolvimento deste trabalho o autor recebeu auxílio
			financeiro da CAPES/CNPq/FAPESP}
	}
	
	\vskip 0.5cm
	\normalsize{
		\br{São Paulo, fevereiro de 2011}
		\en{São Paulo, fevereiro de 2011}
	}
	
\end{center}
\antes{
	% ---------------------------------------------------------------------------- %
	% Página de rosto (SÓ PARA A VERSÃO DEPOSITADA - ANTES DA DEFESA)
	% Resolução CoPGr 5890 (20/12/2010)
	%
	% IMPORTANTE:
	%   Coloque um '%' em todas as linhas
	%   desta página antes de compilar a versão
	%   final, corrigida, do trabalho
	%
	%
	\newpage
	\thispagestyle{empty}
	\begin{center}
		\vspace*{2.3 cm}
		\textbf{\Large{Título do trabalho a ser apresentado à \\
				CPG para a dissertação/tese}}\\
		\vspace*{2 cm}
	\end{center}
	
	\vskip 2cm
	
	\begin{flushright}
		Esta é a versão original da dissertação/tese elaborada pelo\\
		candidato (Nome Completo do Aluno), tal como \\
		submetida à Comissão Julgadora.
	\end{flushright}
}{
	% ---------------------------------------------------------------------------- %
	% Página de rosto (SÓ PARA A VERSÃO CORRIGIDA - APÓS DEFESA)
	% Resolução CoPGr 5890 (20/12/2010)
	%
	% Nota: O título para as dissertações/teses do IME-USP devem caber em um 
	% orifício de 10,7cm de largura x 6,0cm de altura que há na capa fornecida pela SPG.
	%
	% IMPORTANTE:
	%   Coloque um '%' em todas as linhas desta
	%   página antes de compilar a versão do trabalho que será entregue
	%   à Comissão Julgadora antes da defesa
	%
	%
	\newpage
	\thispagestyle{empty}
	\begin{center}
		\vspace*{2.3 cm}
		\textbf{\Large{Título do trabalho a ser apresentado à \\
				CPG para a dissertação/tese}}\\
		\vspace*{2 cm}
	\end{center}
	
	\vskip 2cm
	
	\begin{flushright}
		Esta versão da dissertação/tese contém as correções e alterações sugeridas\\
		pela Comissão Julgadora durante a defesa da versão original do trabalho,\\
		realizada em 14/12/2010. Uma cópia da versão original está disponível no\\
		Instituto de Matemática e Estatística da Universidade de São Paulo.
		
		\vskip 2cm
		
	\end{flushright}
	\vskip 4.2cm
	
	\begin{quote}
		\noindent Comissão Julgadora:
		
		\begin{itemize}
			\item Profª. Drª. Nome Completo (orientadora) - IME-USP [sem ponto final]
			\item Prof. Dr. Nome Completo - IME-USP [sem ponto final]
			\item Prof. Dr. Nome Completo - IMPA [sem ponto final]
		\end{itemize}
		
	\end{quote}
}


\pagebreak

\pagenumbering{roman}     % começamos a numerar 